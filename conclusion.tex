\chapter{Заключение}

\section{Выводы}

В ходе данной работы были рассмотрены некоторые из существующих технологий миграции процессов, на данный момент, ни одна из существующих технологий миграции не приспособлена для кросс-архитектурной миграции процессов.

Также в ходе работы изучен динамический транслятор QEMU, в частности, его режим User Mode трансляции, позволяющий виртуализировать окружение отдельного процесса в ОС Linux. QEMU User Mode выступил в качестве динамического транслятора в разработанной технологии миграции.

Реализовано создание образа процесса ОС Linux исключительно в пространстве пользователя, а также восстановление процесса по сохраненному образу. Надо отметить, что разработанное средство нужно рассматривать не как полноценное решение, а как доказательство практической возможности реализации кросс-архитектурной миграции (proof of concept).

Отметим также, что миграция даже минимального набора ресурсов процесса исключительно средствами пространства пользователя достаточно проблематична, особенно, при восстановлении процесса (см. раздел~\ref{nomem}). Реализация полноценной миграции, определенно, требует поддержки со стороны ядра ОС.

На основе проделнной работы, можно сделать вывод, что при реализации кросс-архитектруной миграции трансляция исполнямого кода не является основным препятствием и может быть достаточно надежно организована существующими средствами, основные же сложности, как и при обычной миграции связаны с сохранением и восстановлением ресурсов процесса.

Также в ходе данной работы произведены измерения потерь производительности при использовании динамической трансляции. Потеря производительности достаточно серьезная, например, десятикратное падение скорости на арифметических операциях (см. раздел~\ref{results}), что, конечно, нельзя назвать приемлимым, особенно, по сравнению с существующими технологиями паравирутализации, которые совершенно не приспособленны для кросс-архитектурной миграции (как, собственно, и виртуализации), но при этом обеспечивают производительность близкую к нативной.~\footnote{http://www.ixbt.com/cm/virtualization-xen.shtml}

\section{Направления развития}

С учетом проделанной работы можно заключить, что основной интерес для дальнейшего развития представляет увеличение производительности динамической трансляции нативного кода. QEMU проектировался как быстрый динамический транслятор, но использование честной (straightforward) трансляции с минимальными оптимизациями не способно обеспечить достаточную для практического применения производительность.

Другим важным направлением развития может стать разработка новой модели памяти. Существующая схема трансляции из-за нерационального использования адресов пораждает проблему нехватки памяти, что серьезно ограничивает класс переносимых программ.

Немаловажным является рассмотрение альтернативных подходов к миграции процессов, например, миграция процессов виртуальных машин Java и .NET.
