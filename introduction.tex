\chapter{Введение}

\section{Энергопотребление процессоров и виртуализация}

Продолжительность работы от батареи является важной характеристикой мобильного устройства. Однако не менее, а, возможно, даже более важным является вопрос энергопотребления и для серверных систем. Энергопотребление большого дата-центра может создавать трудности энергосистемам целых регионов, не говоря уже о затратах на оплату этой электроэнергии.

Например, компания Facebook строит дата-центр в округе Крук штата Орегон. Мощность дата-центра составляет порядка 28 МВт~\footnote{http://www.datacenterknowledge.com/archives/2012/01/30/facebook-has-spent-210-million-on-oregon-data-center/}, при этом потребление всего региона Крук составляет порядка 30 МВт~\footnote{http://gizmodo.com/5880804/facebooks-oregon-data-center-uses-as-much-power-as-the-entire-county}. А суммарную мощность всех дата-центров компании Google в 2011 году оценивали в 220 МВт.~\footnote{http://www.computerra.ru/23033/data-tsentryi-google-facebook-i-apple-potreblyayut-bolshe-e/}

В связи с этим технологические компании начинают проявлять интерес к альтернативной энергетике~\footnote{http://www.slashgear.com/apple-plans-solar-power-for-data-center-26191056/} и сберегающим технологиям. Спектр используемых технологий затрагивает все части современной компьютерной техники:

\begin{itemize}

    \item различные источники энергии (солнечные батареи и ветряки)
    \item энергоэффективные системы хранения данных (SSD)
    \item энергоэффективные процессоры (intel lv и ulv, arm)
    \item виртуализация для более эффективной утилизации аппаратных ресурсов

\end{itemize}

Процессоры intel серии lv и ulv являются энергоэффективными модификациями cisc процессоров семейства x86. Развитие процессоров x86 всегда было нацелено на производительность, альтернативный подход представляют процессоры семейства arm, которые с самого начала были нацелены на минимальное энергопотребление. В связи с развитием мобильной техники процессоры arm стали одними из самых популярных, но теперь же планируется использовать их еще и в серверных системах.~\footnote{http://www.extremetech.com/computing/150664-calxedas-first-arm-server-is-a-serious-threat-to-x86-server-domination}~\footnote{http://www.anandtech.com/show/6418/amd-will-build-64bit-arm-based-opteron-cpus-for-servers-production-in-2014}

В связи с этим актуальным становится вопрос абстракции программного обеспечения от аппаратной платформы. В области построения аппаратно-независимого программного обеспечения существуют широко применимые решения - виртуальные машины JVM и .NET позволяют полностью абстрагироваться от архитектуры процессора и даже, частично, от операционной системы, для нативных приложений также существуют решения, например, QEMU.

Но хочется не просто уметь запускать приложение на любой платформе, но и уметь переносить уже запущенное приложение на другую платформу, т. е. осуществлять кросс-архитектурную миграцию процессов.

\section{Миграция процессов}

Миграция процессов объемный термин. В различных контекстах он может обозначать различные понятия, в данной работе используется следующее простое определение:

\begin{Def}\label{migration}
Миграция процесса - перенос исполняемого процесса и связанных с ним ресурсов с одной вычислительной системы на другую с сохранением состояния процесса. Переносимый процесс будем называть мигрируемым.
\end{Def}

Миграция процессов может быть классифицирована по следующим признакам:

\begin{itemize}

    \item требуется ли поддержка со стороны мигрируемого процесса
    \item необходим ли останов мигрируемого процесса

\end{itemize}

\begin{Def}\label{live_migration}
Если во время миграции не требуется останов мигрируемой системы, или этот останов не требует поддержки самой системы и не заметен для нее, тогда миграция называется живой.
\end{Def}

\begin{Def}\label{seamless_live_migration}
Если останов мигрируемой системы системы незаметен для внешнего наблюдателя, например, пользователя сервиса, такая миграция называется бесшовной.
\end{Def}

\begin{Def}\label{process_image}
Образ системы - набор данных, по которым система может быть восстановлена в определенном состоянии.
\end{Def}

\subsection{Применение миграции}\label{migration_applications}

\textit{Довести до 1 страницы, без учета списков, нужны более конкретные примеры}

Поддержка миграции достаточно сложная задача, однако существует множество реализаций миграции внутри различных гипервизоров~\footnote{http://en.wikipedia.org/wiki/Live\_migration}:

\begin{itemize}

    \item Virtuozzo
    \item Xen
    \item OpenVZ
    \item Workload Partitions
    \item Integirity Virtual Machines
    \item KVM
    \item Oracle VM
    \item POWER Hypervisor
    \item VMware ESX
    \item IBM VPAR
    \item Hyper-V Server 2008 R2
    \item VirtualBox

\end{itemize}

Также существуют самостоятельные утилиты позволяющие мигрировать отдельные процессы в рамках различных операционных или распределенных систем:

\begin{itemize}

    \item CRIU
    \item Linux-cr
    \item DMTCP
    \item BLCR
    \item OpenSSI
    \item MOSIX

\end{itemize}

Но все перечисленные системы либо узко направлены как, например, OpenSSI, либо не имеют поддержки кросс-архитектурной миграции, как CRIU или миграция внутри гипервизоров.

Применение технологий миграции процессов достаточно обширно~\cite{criu_yandex_presentation}:

\begin{itemize}

    \item ускорение старта тяжелых приложений
    \item снимки состояния рабочей системы
    \item балансировка нагрузки в распределенной вычислительной системы
    \item создание точек возврата (резервных копий)
    \item для повышения надежности сервисов

\end{itemize}

\subsection{Кросс-архитектурная миграция}

\begin{Def}\label{cross_migration}
Кросс-архитектурная миграция - миграция между системами различной аппаратной архитектуры.
\end{Def}

Сложность реализации кросс-архитектурной миграции заключается в том, что состояние мигрируемого процесса зависит от архитектуры, а следовательно должно подвергаться преобразованию при восстановлении процесса, например, список регистров различается от процессора к процессору.

Кроме того необходимо транслировать исполняемый код исходной архитектуры в исполняемый код целевой, а это не всегда возможно без интерпретации этого кода (т. е. фактически без его исполнения), так как данные и код программы в памяти не отличимы до момента исполнения, не говоря уже о том, что программа может генерировать код на лету, как это делают Java и .NET виртуальные машины, или, например, QEMU.

В данной работе я разработал технологию кросс-архитектурной миграции процессов ОС Linux на основе динамической трансляции не требующую поддержки со стороны ядра операционной системы, т. е. работающую в пространстве пользователя.
