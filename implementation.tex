\chapter{Технология кросс-архитектурной миграции}

\section{Верхнеуровневая архитектура}

\textit{1.5 страницы}

Картинка основных компонентов системы. Описание назначения основных компонентов системы. Используемые интрефейсы операционной системы - ptrace и proc.

\section{Останов процесса}

\subsection{Архитектура}

\textit{1 страница}

Картинка архитектуры приложения. Выделение архитектурно зависмых и архитектурно независимых частей приложения.

\subsection{Системный вызов ptrace}

\textit{> 0.5 страницы}

Вызов ptrace для останова процесса. Получение состояния регистров. eax и orig\_eax. Аппаратно нехзависимое представление регисров и интерфейс для преобразования аппаратно зависимого представления в аппаратно незаивсимое представление. Формат хранения регистра.

\subsection{Файловая система proc}

\textit{> 0.5 страницы}

Файловая система ptrace для доступа к памяти процесса. Формат сохранения памяти процесса.

\subsection{Организация проекта и система сборки}

\textit{0.5 страницы}

Организация директорий проекта. Зависимости проекта. Генерация файлов protobuf по файлам описания

\section{Восстановление процесса}

\subsection{Организация QEMU и система сборки}

\textit{1 страница}

Основные компоненты транслятора QEMU. User Mode режим.

\subsection{Восстановление памяти гостевого процесса}

\textit{1 страница}

Отображение адресов гостя в адреса хоста (картинка). Проблема нехватки адресного пространства и ее "решение". Интерфейсы QEMU для резервирования памяти. Чтение файлов protobuf.

\subsection{Восстановление регистров процессора}

\textit{0.5 страницы}

Виртуальный процессор QEMU (вырезка из структуры). Чтение файлов protobuf.
